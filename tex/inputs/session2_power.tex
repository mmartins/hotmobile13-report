%!TEX root = ../hotmobile13-report.tex
\section{Power Management}
\label{sec:power}
The Power Management session, chaired by Nigel Davies (Lancaster University)
began with ``How is Energy Consumed in Smartphone Display Applications?''. 
Presented by Mian Dong (Samsung Telecommunication America), the work explored
the power consumption of smartphone AMOLED screens. Their first 
research question focused on improvements to AMOLED screens over time
%(\eg{} the Super AMOLED on the Nexus S [2010] vs. the HD Super AMOlED Plus on 
%the Galaxy S III [2012])
and found that the practical power improvement is relatively low.
%(between 6\% and 30\% in the best conditions)
A second study focused on 
producing a power analysis for the display in an applications context. For three 
sample applications: video, gameplay, and camera, the display was found to be a 
relatively small contributor to power consumption. However, some content may lead 
to very high power consumption which could be reduced using colour 
transformation (\eg{} reducing brightness, increasing saturation).

Mostafa Uddin (Old Dominion University) then presented ``A2PSM: Audio Assisted 
Wi-Fi Power Saving Mechanism for Smart Devices''. Wi-Fi power saving mechanisms 
(PSMs) are key for the extending the limited power resources of mobile devices. 
A2PSM exploits lower-powered audio hardware (\eg{} microphone\slash speaker) to 
improve existing PSMs (\eg{} static PSM) by reducing the need for `wake periods'. 
Evaluation of a smartphone prototype suggests A2PSM could offer in excess of 25\%
improvement over SPSM with no impact on network throughput.

The final paper of the session, ``Application Modes: A Narrow Interface for End-
User Power Management in Mobile Devices'' was presented by Marcelo Martins (Brown 
University). The work suggests `Application Modes' as a mechanism for allowing 
the user to make power management decisions on their mobile device. Each 
application can offer a number of modes that comprise of reductions of 
functionality with associated power savings (\eg{} turn off automatic 
synchronisation). Modes allow developers to provide graceful degradation of their 
application in resource-constrained circumstances, and also allow the user to 
prioritise the application they are using over less-important ones.