%!TEX root = ../hotmobile13-report.tex
\section{Power Management}
\label{sec:power}
The session %(chaired by Nigel Davies, Lancaster University)
opened with
``How is Energy Consumed in Smartphone Display Applications?'' presented by Mian Dong (Samsung Telecommunication America). The work explored
the power consumption of smartphone AMOLED screens through two research questions: the first focused on improvements to AMOLEDs over time
and found that the practical power improvement is relatively low.
The second question focused on 
producing a power analysis for the display in an application context. For three 
sample applications (video, gameplay, camera), the display was found to be a 
relatively small contributor to power consumption. However, some content can result in high power consumption, which may be reduced using colour 
transformation (\eg{} reducing brightness, increasing saturation).

Mostafa Uddin (Old Dominion University) then presented ``A2PSM: Audio Assisted 
Wi-Fi Power Saving Mechanism for Smart Devices''. Wi-Fi power saving mechanisms 
(PSMs) are key for the extending the limited power resources of mobile devices. 
A2PSM exploits lower-powered audio hardware to 
improve existing PSMs by reducing the need for `wake periods'. 
Evaluation of a smartphone prototype suggests A2PSM could offer in excess of 25\%
improvement over SPSM with no impact on network throughput.

The final paper of the session, ``Application Modes: A Narrow Interface for End-
User Power Management in Mobile Devices'' was presented by Marcelo Martins (Brown 
University). `Application Modes' are suggested as a mechanism for allowing 
users to make power management decisions on their mobile device. Each 
application can offer a number of modes that comprise of reductions of 
functionality with associated power savings (\eg{} turn off automatic 
synchronisation). Modes allow developers to provide graceful degradation of their 
application in resource-constrained circumstances, and also allow the user to 
prioritise an in-use application over background applications.

At the close of this session, discussion focused around the idea
that mobile computing has potentially reached a point where all easy 
optimisations are gone; further energy improvements must impact the 
user. Offloading decisions to the user is difficult and it is difficult to 
know which applications are draining battery. Motivating 
developers to conserve energy, and raising awareness of application energy use, 
could be targeted in future work. %Users could potentially balance energy (and 
%other costs) against the initial cost of an application to calculate which 
%applications were truly the most explsnive.