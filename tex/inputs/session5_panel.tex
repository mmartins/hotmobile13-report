%!TEX root = ../hotmobile13-report.tex
\section{Panel -- Mobile Systems and the Developing World}
\label{sec:panel}

Gaetano Borriello (University of Washington) moderated the panel on
Information and Communication Technology for Development (ICTD) applied
to mobile computing. Recent developments in mobile computing and
communication led to the proliferation of mobile phones worldwide.
Thanks to their lower prices, these devices have become the main
platform access to technology in emerging markets, substituting
notebooks and desktop computers. Three academic experts provided their
input based on their research experience.

Elizabeth Belding (UCSB) summarised some of her technological
partnerships with African communities to bring wireless Internet
connectivity and services to small villages. Belding introduced her work
with UCSB students on devising mechanisms for sharing traffic and
content using wireless technology and reducing costs of cellular and
data telephony in remote locations. Lakshmi Subramanian (New York
University) gave the audience a glimpse of his research on mobile
solutions to bridge the digital divide: a web search engine based on SMS
input, keeping connectivity in unstable rural cellular networks, among
others. Subramanian's attention and adaptation to
resource constraints was particularly impressive.
Bill Thies (MSR India) continued the series of
``design for extremes'' by showing some of his creativity to bring
research opportunities using new solutions for constrained environments.
In special, Thies showed a P2P system that his team at MSR designed to
share information downloaded from the Web with peers when there is very
limited connectivity.

The discussion following the panelists' speeches focused on the scope
and feasibility of ICTD as a research area. The panelists noted that the
IT disparage does not only exist because of economical problems, but
also due to cultural issues that affect technological absorption. They
also noted that community efforts should focus on problems that will
linger for many generations. Finally, the panelists declared that
research on ICTD is a gratifying experience, as they create
opportunities for local training and local employment by start-up
companies.
