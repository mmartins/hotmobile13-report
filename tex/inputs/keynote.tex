%!TEX root = ../hotmobile13-report.tex
\section{Keynote -- Wearable Computing: Through the Looking Glass}
\label{sec:keynote}
Mobile computing systems could easily be accused of consuming too much
attention, diverting concentration from the physical environment. Thad
Starner's opening keynote highlighted the contrast between `mobile'
computing and true `on-the-go' computing -- devices you can really use in
motion. Starner suggests the use of `micro-interactions' to reduce the time 
between intention and action, allowing truly mobile computing. Throughout the 
keynote, Starner used examples from his labs to demonstrate the potential of 
micro-interactions.

%One early example from Starner's labs, `the Twiddler' provided an alternative to
%traditional mobile keypads and keyboards by facilitating desktop keyboard-level
%typing rates %(40-130 wpm as opposed to the 8-20 wpm seen by skilled T9 users)
%on a single-handed keyboard device with only a minimal learning curve
%(approximately three times faster than a desktop keyboard).

One early (pre-Smartphone) example allowed Starner to conduct Web searches 
via his wristwatch, typically demonstrated as a `parlour trick' in which 
he offered to answer any question posed. Recently, Starner has been involved in 
Google's Project Glass. Playing a recent promotional video 
\footnote{\url{http://www.google.com/glass/start/how-it-feels/}}, Starner showed 
Glass's potential to support micro-interactions for video and still 
photography, and voice chat, amongst others.

Another example, the Mobile Music Touch (MMT) is a wireless glove
with vibration motors in the fingers. Wearing the glove allows an individual to
learn piano fingering whilst attending to other tasks.
Trials of MMT show effective learning whilst completing exams, presenting a
conference paper and forecasting the weather
\footnote{\url{http://www.youtube.com/watch?v=uEdr6iY6F-w}}.
In the healthcare domain,
MMT has been shown to be an effective tool in tetraplegic rehabilitation; 
patients showed increased mobility and sensation, whilst the learning of a life-
long skill provided incentive to continue.

% Questions
MMT dominated the audience questions. Degradation was one concern -- Starner 
reported that six months after their rehabilitation study, one participant played 
a piece for CNN -- just fifteen minutes practice was sufficient to refresh 
learning. Exploration of the limits and alternative applications is also an 
interesting area.

Jason Hong (Carnegie Mellon University) asked about insights gained from using
the various wearable technologies. Starner responded by noting that 
``almost everyone does not use a technology the way they initially 
think they will'' and that productivity tools often emerge -- perhaps wearable 
computing is less about the ``killer app'' and more about the killer 
\emph{lifestyle}.
