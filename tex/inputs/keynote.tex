%!TEX root = ../hotmobile13-report.tex
\section{Keynote -- Wearable Computing: Through the Looking Glass}
\label{sec:keynote}
Current mobile computing systems could easily be accused of consuming too much
of our attention, diverting concentration from the physical environment. Thad
Starner's opening keynote highlighted the contrast between this so-called
mobile computing and true `on-the-go' computing -- devices you can really use in
motion. Starner suggests the use of `micro-interactions' to reduce the time 
between intention and action and allows truly mobile computing. Throughout the 
keynote, Starner used examples from his labs to demonstrate the potential of such 
micro interactions.

%One early example from Starner's labs, `the Twiddler' provided an alternative to
%traditional mobile keypads and keyboards by facilitating desktop keyboard-level
%typing rates %(40-130 wpm as opposed to the 8-20 wpm seen by skilled T9 users)
%on a single-handed keyboard device with only a minimal learning curve
%(approximately three times faster than a desktop keyboard).

One early (pre-Smartphone) example allowed Starner to search the Web through
his wristwatch, typically demonstrated through a 'parlour trick' in which he
offered to answer any question posed to him -- as long as the answer was given
within the first few Web results!
Recently Starner has been a Technical Lead\slash Manager on Google's Project 
Glass. Playing a recent Glass video promo
\footnote{http://www.google.com/glass/start/how-it-feels/}, Starner showed how 
the device had the potential to support micro interactions for video and camera 
photography and voice chat amongst others.

Another example, The Mobile Music Touch (MMT) takes the form of a wireless glove
with vibration motors in the fingers. Wearing the glove allows an individual to
learn piano fingering for a musical piece whilst attending to other tasks.
Trials of the MMT show effective learning whilst completing exams, presenting a
conference paper and forecasting the weather
\footnote{http://www.youtube.com/watch?v=uEdr6iY6F-w}. In the healthcare domain,
the MMT has been shown to be an effective tool in the rehabilitation of
tetraplegic patients; study participants showed increased mobility, increased
sensation and had additionally begun the learning of a life-long skill -- an
incentive to continue with the rehabilitation.

% Questions
Starner's Mobile Music Touch gloves dominated the audience questions.
Degradation was one concern raised by the audience -- Starner reported that six
months after their eight-week rehabilitation study, one of the participants
played their piece for CNN, just fifteen minutes practice with the glove was
sufficient to refresh the learning. Exploration of the limits and alternative
applications for the technology is also an interesting area -- for example,
could the technology be used to improve the 90\% drop out rate for
stenographers?

Towards the beginning of his talk Starner had suggested that at least one
month's use of a new wearable device was needed in order to discover the
qualitative differences of using the technologies. Jason Hong (Carnegie
Mellon University) asked about the insights gained through such a period.
Starner suggested that the key issue here was that almost everyone does not 
use a technology the way they initially think they will and productivity tools
often emerge -- perhaps wearable computing is less about the ``killer app'' and
more about the killer \emph{lifestyle}.