%!TEX root = ../hotmobile13-report.tex
\section{Understanding Humans}
\label{sec:humans}

This session, chaired by Anthony LaMarca (Intel Labs), opened with Xuan
Bao (Duke University) presenting ``The Case for Psychological
Computing''. Bao advocated a system capable of modelling psychological
attributes as inherent nature (long-term, fixed values) and states
(short-term, context-dependent values). He believes that developers can
make use of these variables to create applications that better meet
human properties, \eg, preferential content pre-fetching and vehicle
driving with cognitive comfort. Bao concluded his talk by mentioning
some of the remaining challenges for enabling psychological computing.
These include how to model attributes correctly, and in case of wrong
prediction, how to back off gracefully.

He Wang (Duke University) followed with ``Recognizing Humans without Face
Recognition'', a presentation of the \emph{InSight} system for human
identification using camera-enabled portable devices in situations where
computer vision cannot distinguish face patterns. Wang found that
applying spectogram analysis to cloth colors and wavelets to cloth
patterns are prominent techniques for accurately recognising humans from
their garment. InSight can share these fingerprints with other people's
devices to enable social-aware applications.

Shahriar Nirjon (University of Virgina) presented the last paper of
the session, ``sMFCC: Exploiting Sparseness in Speech for Fast Acoustic
Feature Extraction on Mobile Devices -- a Feasibility Study''. Nirjon
noted that current solutions for voice-driven smartphone applications
must offload the extraction of smartphone audio features to the cloud to
guarantee real-time system response, which leads to high communication
cost. After empirically observing the sparseness of frequency domain in
speech, Nirjon proposed sMFCC (sparse Mel-Frequency Cepstral
Coefficients) to exploit this property and enable in-situ, fast
acoustic-feature extraction without relying on the cloud. Nirjon built
sMFCC on top of recent work on sparse Fast Fourier Transform (sFFT) as
an efficient approximation to the widely used MFCC algorithm. He showed
that sMFCC can achieve an accuracy of up to $83.97\%$ on word
recognition with a low expected response time.

Privacy was the key topic during the discussion session, raising
concerns on unwanted disclosure of personal data, especially in the case
of both presentations from Duke. The main argument laid on whether a
user would be willing to expose herself in exchange of functionality and
what the system could do with the collected information about the user.
