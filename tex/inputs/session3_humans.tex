\section{Understanding Humans}
\label{sec:session3_humans}

This session opened with Xuan Bao of Duke University presenting ``The
Case for Psychological Computing.'' His team work envisions the pushing
of psychological factors into deeper stacks of system design to
intimately understand human needs and design better computing solutions.
This novel system models psychological attributes as inherent nature
(long-term, fixed values) and states (short-term, context-dependent
values).  Bao believes that developers can make use of these variables
to create applications that better meet human properties, such as
preferential content pre-fetching and vehicle driving with cognitive
comfort. Remaining challenges include how to model these values
correctly, and in case of wrong prediction, how to back off graciously.

He Wang, also from Duke, followed with ``Recognizing Humans without Face
Recognition'', a presentation of the \emph{Insight} system for human
identification using camera-enabled portable devices in situations where
Computer Vision cannot be applied to face patterns. Wang found that
spectogramming applied to cloth colors and waveletting over cloth
patterns are prominent techniques for recognizing humans using their
vestments with great accuracy. InSight shares these fingerprints with
other people's devices to enable social-aware applications.

Shahriar Nirjon, from University of Virgina, presented the last paper of
the session, ``sMFCC: Exploiting Sparseness in Speech for Fast Acoustic
Feature Extraction on Mobile Devices -- a Feasibility Study''. Nirjon
pointed out that current solutions for voice-driven smartphone
applications must offload the extraction of frequency-domain acoustic
features to the cloud. This is necessary to guarantee near real-time
system response, but incurs a high communication cost. sMFCC (sparse
Mel-Frequency Cepstral Coefficients) exploits the empirically observed
sparseness of frequency domain in speech to enable fast extraction of
the same features from high-sampling-rate audio running solely on a
smartphone. Nirjon built sMFCC on top of recent work on sparse Fast
Fourier Transform (sFFT) as an efficient approximation of the widely
used MFCC features on the phone. His proposal achieves an accuracy of up
to $83.97\%$ on word recognition with expected computation time lower or
equal to the duration of speech samples.

\remark{
  \textbf{TODO}:
  \begin{itemize}
  \item Need for sFFT reference?
  \item Add Discussion session.
  \end{itemize}
}
