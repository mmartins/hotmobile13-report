%!TEX root = ../hotmobile13-report.tex
\section{Understanding Humans}
\label{sec:humans}

This session opened with Xuan Bao of Duke University presenting ``The
Case for Psychological Computing.'', a proposal for pushing
psychological factors into deeper stacks of system design to intimately
understand human needs and design better computing solutions.  Bao
advocates a system capable of modelling psychological attributes as
inherent nature (long-term, fixed values) and states (short-term,
context-dependent values). He believes that developers can make use of
these variables to create applications that better meet human
properties, \eg, preferential content pre-fetching and vehicle driving
with cognitive comfort. Bao concluded his talk by mentioning some of the
remaining challenges for enabling psychological computing. These include
how to model attributes correctly, and in case of wrong prediction, how
to back off graciously.

He Wang, also from Duke, followed with ``Recognizing Humans without Face
Recognition'', a presentation of the \emph{InSight} system for human
identification using camera-enabled portable devices in situations where
computer vision cannot work with face patterns. Wang found that
applying spectogram analysis to cloth colors and wavelets to cloth
patterns are prominent techniques for accurately recognizing humans from
their vestments. InSight shares these fingerprints with other people's
devices to enable social-aware applications.

Shahriar Nirjon, from University of Virgina, presented the last paper of
the session, ``sMFCC: Exploiting Sparseness in Speech for Fast Acoustic
Feature Extraction on Mobile Devices -- a Feasibility Study''. Nirjon
noted that current solutions for voice-driven smartphone applications
must offload the extraction of acoustic features to the cloud to
guarantee near real-time system response, despite a high communication
cost. After empirically observing the sparseness of frequency domain in
speech, Nirjon proposed sMFCC (sparse Mel-Frequency Cepstral
Coefficients) to exploit this property and enable in-situ, fast
acoustic-feature extraction without relying on the cloud. Nirjon built
sMFCC on top of recent work on sparse Fast Fourier Transform (sFFT) as
an efficient approximation to the widely used MFCC algorithm. He showed
that sMFCC can achieve an accuracy of up to $83.97\%$ on word
recognition with a low expected response time.

\remark{
  \textbf{TODO}:
  \begin{itemize}
  \item Need for sFFT reference?
  \item Add Discussion session.
  \end{itemize}
}
