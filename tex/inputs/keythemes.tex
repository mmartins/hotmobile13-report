%!TEX root = ../hotmobile13-report.tex
\section{Reoccurring Themes}
\label{sec:keythemes}

% Improving user experience
Throughout the workshop the issue of user experience frequently generated 
discussion. In the keynote (\S\ref{sec:keynote}), Starner identified a need 
to improve interaction such that mobile computing reduced attention and cognitive 
demands whilst also highlighting that use of technology rarely resembled initial 
predictions. Concerns about cognitive load were raised again within the Vehicular 
Networking session (\S\ref{sec:vehiclenets}) where the system `Scout' posed 
a solution that, in turn, brought its own loading issues. Within the mobile 
billing and energy sessions (\S\ref{sec:cellbilling} and \S\ref{sec:power} 
respectively), balancing customisation effort against optimisation gains was a 
key concern, as was making clear visualisations of energy\slash network use such 
that user could make appropriate decisions. One potential technique for 
mitigating resource constraints was raised in the Understanding Humans session 
(\S\ref{sec:humans}) through Bao et al.'s suggestion of integrating
psychological factors into systems to improve user experience. The
session on mobile cloud interactions (\S\ref{sec:mobilecloud}) embraced
experimental work on seamless user interaction with different mobile
devices, shared programmable displays and mobile Internet.

% Enabling new applications
Another frequent discussion topic was the empowerment of new
applications. Within the panel on mobile computing for the developing
world (\S\ref{sec:panel}), many presented solutions used the
technological constraints as a motivator for development of creative and
affordable solutions to bridge the IT gap in small communities. The
session on sensors and data (\S\ref{sec:sensors}) also showed a variety
of groundwork to facilitate the development of mobile applications,
be it from the perspective of collaborative sensing, data-collection
solutions, or privacy-enhancing techniques.

% \ref{sec:mobilecloud}
% \ref{sec:panel}
% \ref{sec:sensors}
% \ref{sec:postersdemos}
