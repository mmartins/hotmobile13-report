%!TEX root = ../hotmobile13-report.tex
\section{Sensors and Data}
\label{sec:sensors}

Rajesh Krishna Balan (Singapore Management University) chaired the
session. Yu Xiao (CMU/Aalto University) started her talk, ``Lowering the
Barriers to Large-Scale Mobile Crowdsensing'', by listing the main
obstacles to a successful scaling of collaborative sensing on mobile devices:
device heterogeneity, lack of user incentives and high communication
cost. She urged the community to rethink the deployment model for this
type of applications and suggested a three-tier architecture to ease
scaling. She proposed the use of a middle layer between mobile clients
and app server, named \emph{cloudlets}. Cloudlets work as a
distributed infrastructure for running VMs, physically close to the
mobile client to reduce networking costs. Cloudlets comprise a set of
proxy VMs, which abstract device heterogeneity and handle requests of
sensor data on behalf of mobile clients, and application VMs, which
decentralize application logic and data aggregation. She explained that
the app server manages the transfer of VMs between cloudlets according
to the user's position, to guarantee both low latency and logic
offloading. She concluded her presentation with some technical
challenges to enabling this new technology: virtualization overhead and
cloudlet reconfiguration due to VM migration.

Waylon Brunnete (University of Washington) followed with ``Open Data Kit
2.0: Expanding and Refining Information Services for Developing
Regions''. Brunnette elaborated on the success of Open Data
Kit~\footnote{\url{http://opendatakit.org}} (ODK) 1.0 as a modular
toolkit for building mobile applications mainly used for data collection
and aggregation in developing regions. From the feedback received from
various developers worldwide, Brunnette iterated over the drawbacks of
ODK 1.0 and completed his talk by explaining how his team at UW fixed
these shortcomings in ODK 2.0 and enabled the creation of more complex
applications.

%in special the adoption of databases
%instead of forms to enable editing of collected data, preference of
%runtime languages over compiled ones to modify apps without recompiling
%and support for more built-in and external sensors to collect data from
%various sources.

Supriyo Chakraborty (UCLA) concluded the session with ``A
Framework for Context-Aware Privacy of Sensor Data on Mobile Devices'',
a study on the tradeoff between application utility and shared-data
privacy. Chakraborty argued about the risk of disclosing sensitive
information by untrusted apps via inferences on shared personal
data. To preserve anonymity and avoid private-data reconstruction, the
presenter hinted on converting data to feature spaces with lower
dimensions before sharing and partially obfuscating these features by
sharing only what is explicitly vetted by the user. Chakraborty finished
his talk by describing \emph{ipShield}, an inference privacy framework
implemented atop Android that infers the adversary capability of
extracting sensitive information from shared data, and provides a
firewall for coping with data-sharing filters for apps and data
obfuscation rules by users.
