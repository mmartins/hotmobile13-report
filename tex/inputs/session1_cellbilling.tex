%!TEX root = ../hotmobile13-report.tex
\section{Cellular Billing}
\label{sec:cellbilling}

The first session was chaired by Suman Banerjee (University of Wisconsin-Madison)
and began with ``Splitting the Bill for Mobile Data with SIMlets'' presented by 
Alec Wolman (Microsoft Research). Motivated by the limited mobile broadband 
infrastructure, the paper suggests the use of split billing. Split billing allows 
content providers to pay costs generated by mobile users using their services.
trustworthy abstraction whose policy indicates which traffic should be billed to 
which entity. Content providers issue a SIMlet to a mobile device in order to 
specify which data they are prepared to pay for.
Audience members suggested that mobile operators might also be incentivised 
towards split billing, but that the key challenge is the development of the 
required trust relationships. Jeff Bulling (FIXME) also questioned the 
scalability of the hardware-specific aspects of their approach.

Younghwan Go (KAIST) then presented ``Towards Accurate Accounting of Cellular 
Data for TCP Retransmission''. At the transport level, application data
is augmented by headers and subject to retransmissions. In 
poor infrastructure conditions, retransmissions can greatly increase the traffic 
exchanged. The authors show that mobile billing policies vary; some charge users 
for retransmissions whilst others allow `free-riding' through abuse of transport 
protocols. By highlighting potential attacks, and suggesting mitigation 
techniques, the authors hope that future billing techniques might accurately 
reflect a user's application use.
The aspiration towards perfect accounting attracted attention during audience
questions -- the suggestion of ``unlimited'' data plans as an
alternative to accurate billing was made. Thad Starner also highlighted the 
energy cost of retransmissions % (for already restricted mobile users)
and the opportunity for such retransmission attacks to potentially jam an already 
limited mobile data service.
