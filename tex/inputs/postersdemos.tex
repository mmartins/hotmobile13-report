%!TEX root = ../hotmobile13-report.tex
\section{Posters \& Demos}
\label{sec:postersdemos}
Winner of the best demo award, ``Bringing In-situ Social Awareness to Mobile 
Systems: Conversational Turn Monitoring and its Applications'' (Lee et al., 
KAIST) attracted significant attention and discussion from attendees. Their 
system used the smartphone as a mechanisms for encouraging in-person interactions
by monitoring conversation patterns. The demo was supported by a video of the 
system in use in a nursery -- encouraging peer-engagement from an otherwise 
withdrawn autistic child. By combining the video with a layout of smartphones 
running their turn monitoring system (which used coloured bars to indicate 
speakers and conversation duration), attendees were able to see a clear 
demonstration of the capability of the system as well as potential applications
and were rewarded for their interactions with the demonstrators with a clear
representation of their conversation patterns.

``Leveraging Imperfections of Sensor for Fingerprinting Smartphones''
(Dey et al., University of South Carolina) received an honorable mention
as the second poster/demo of HotMobile'13. Given the subtle
imperfections of built-in sensor chips resultant of the manufacturing
process, smartphones could be distinguished with high accuracy by comparing
the sensor readings obtained from the same stimulus. Considering that
smartphone IMEI codes can be counterfeit, sensor fingerprinting seems
a promising alternative for telling devices apart.

% Best poster:

% Poster runner-up: ``Leveraging Imperfections of Sensors for Fingerprinting Smartphones''
