%!TEX root = ../hotmobile13-report.tex
\section{Mobile Cloud Interactions}
\label{sec:mobilecloud}

This session, chaired by Matt Welsh (Google), opened with Lars Tiede
(University of Tromsø) presenting ``Cloud Displays for Mobile Users in a
Display Cloud''. Tiede introduced the cloud display, an abstraction for
public, shared programmable displays which can be discovered, composed
and configured by mobile devices. Visual output comes from
applications running on desktop computers, while mobile devices serve as
advanced remote controllers. Tiede demonstrated his functioning
prototype based on VNC technology for screen arrangement,
authentication, content display and transfer. He showed that the
overhead generated from the communication between displays and
running-app device is negligible compared to existing network
technology.

Jeffrey Bickford (AT\&T Security Research Center) followed with
``Towards Synchronization of Live Virtual Machines among Mobile
Devices''. His presentation focused on the design of \emph{VMsync}, a
service to enable users to switch seamlessly between mobile devices
with both data and computation states preserved. He explained that VMsync
aims for minimal user-perceived delay during switching by incrementally
transferring checkpoints from one active VM on one device to other
devices. For the rest of his talk, Bickford analysed different
synchronisation policies for representative multimedia mobile workloads.
He observed that sending deltas between checkpoints decreases the
synchronisation latency, and although increasing the checkpoint interval
can help reduce memory and file-system delta sizes, such approach may
risk the promised seamlessness in user experience.

Michael Butkiewicz (UC Riverside) completed the roster with ``Enabling
the Transition to the Mobile Web with WebSieve'', a study on optimising
the load time of websites in mobile devices. Butkiewicz bases his work
on the observation that a great number of popular websites do
not provide an equivalent mobile-friendly version, resulting in high
load times and user-experience deterioration. He proposed
\emph{WebSieve}, an architecture to estimate the user-perceived value of
web objects on a page and, given a loading-time constraint, load only
those objects that have the highest value to the user. Based on user
surveys, Butkiewicz discovered that different users perceive different
utilities for the same set of objects, hence WebSieve necessitates
personalisation.  Finally, he described his prototype design as an
optimisation-problem solver over an object-dependency graph and covered
some of the non-trivial research questions WebSieve brings about.
