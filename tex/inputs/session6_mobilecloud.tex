%!TEX root = ../hotmobile13-report.tex
\section{Mobile Cloud Interactions}
\label{sec:mobilecloud}

This session, chaired by Matt Welsh (Google), opened with Lars Tiede
(University of Tromsø) presenting ``Cloud Displays for Mobile Users in a
Display Cloud''. Tiede introduced the cloud display, an abstraction for
public, shared programmable displays which can be discovered, composed
and configured from mobile devices. Visual output comes from
applications running on desktop computers, while mobile devices serve as
advanced remote controllers. Tiede demonstrated his functioning
prototype based on VNC technology for screen arrangement,
authentication, content display and transfer. He showed that the
overhead generated by the communication between displays and running-app
device is negligible compared to the existing network technology.

Jeffrey Bickford (AT\&T Security Research Center) followed with
``Towards Synchronization of Live Virtual Machines among Mobile
Devices''. His presentation focused on the design of \emph{VMsync}, a
service to enable users to switch seamlessly between mobile devices
with both data and computation state preserved. He explained that VMsync
aims for minimal user-perceived delay during switching by incrementally
transferring checkpoints from one active VM on one device to other
devices. For the rest of his talk, Bickford analyzed different
synchronization policies for representative multimedia mobile workloads.
He observed that sending deltas between checkpoints can decrease the
synchronization latency, and although increasing the checkpoint interval
can help decrease the memory and file system delta sizes for dynamic
workloads, this approach may risk a seamless user experience.

Michael Butkiewicz (UC Riverside) completed the roster with ``Enabling
the Transition to the Mobile Web with WebSieve'', a study on optimizing
the load time of websites in mobile devices. Butkiewicz bases his work
on the observation that a great number of popular complex websites do
not provide a mobile-friendly version, which leads to high load times
and deteriorates user experience. He envisions \emph{WebSieve}, an
architecture to estimate the user-perceived value of each web object of
a page and, given a loading-time constraint, only load those which have
the highest value to the user. Based on user surveys, Butkiewicz
discovered that different users perceive different utilities for the
same set of objects, hence WebSieve necessitates personalization.
Finally, he described his prototype design as an optimization-problem
solver over an object-dependency graph and covered some of the
non-trivial research questions WebSieve brings about.

\remark{TODO: Add discussion}
