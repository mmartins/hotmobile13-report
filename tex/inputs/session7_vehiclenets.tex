%!TEX root = ../hotmobile13-report.tex
\section{Vehicular Networking and Transportation}
\label{sec:vehiclenets}
The final session, chaired by Lin Zhong, opened with the paper ``Scout: An 
Asymmetric Vehicular Network Design over TV Whitespaces'' presented by Tan Zhang 
(University of Wisconsin-Madison).

Motivated by an increasing trend towards in-vehicle Internet connectivity, Scout 
provides an architecture for vehicular networking by leveraging locally available 
television frequencies (whitespaces). Power transmission restrictions suggest an 
asymmetric design so whitespace is used as a downlink whilst cellular 3G is 
used for the uplink. Two radios are used per vehicle, the front (scout) 
monitors changing channel characteristics (\eg{}, the vehicle passing behind a 
building) and informs the base station which then optimises communication with 
the rear radio. Initial evaluation indicates that Scout can improve coverage by 
4$\times$ and increase throughput by 1.4$\times$.

Wenjie Sha (Rutgers University) then presented ``Social Vehicle Navigation: 
Integrating Shared Driving Experience into Vehicle Navigation''. Sha described 
their system \emph{NaviTweet}, a social navigation application which allows drivers 
to post or listen to voice tweets grouped by vehicular social 
network (VSN, a social network of drivers who travel the same routes or to the 
same destination). Voice tweets are formed into digests by the 
NaviTweet server and then sent to driver's mobile device for playback.
Whilst the audio nature of the tweets 
was intended to reduce cognitive overload, audience questions highlighted the 
potential for background noise to make interpretation challenging and even for 
the opportunity for transmission of negative, `road rage', messages.

The final paper, ``Quantifying the Potential of Ride-Sharing using Call 
Description Records'', was presented by Blerim Cici (University of California). 
The work attempts to quantify the upper bound on ride-sharing 
by considering human mobility patterns using a 
3-month call description record (CDR) dataset.
By permitting short detours to pickup\slash drop-off,
analysis shows that ride-sharing has the potential to eliminate ~50\% of car 
journeys. During discussion following Cici's presentation, audience members
considered methods of encouraging ride-sharing by identifying incentives to
share (\eg{}, through use of social graphs) and increasing opportunity by 
allowing passengers divide their route between cars. The potential to
improve public transportation through CDR analysis was also suggested.