%!TEX root = ../hotmobile13-report.tex
\section{Vehicular Networking and Transportation}
\label{sec:vehiclenets}

The final session, chaired by Lin Zhong, opened with the paper `Scout: An 
Asymmetric Vehicular Network Design over TV Whitespaces'' presented by Tan Zhang 
(University of Wisconsin Madison).

Motivated by an increasing trend towards in-vehicle Internet connectivity, Scout 
provides an architecture for vehicular networking by leveraging locally available 
television frequencies (whitespaces). Power transmission restrictions suggest an 
asynchronous design and so whitespace is used as a downlink whilst cellular 3G is 
used for the uplink. By using two radios on each vehicle, the front (scout) can 
monitor changing channel characteristics (\eg{} as the vehicle passes behind a 
building) and inform the base station which can then optimise communication with 
the rear radio. Initial evaluation indicates that using this architecture can 
improve base station coverage by 4x and achieve a throughput gain of 1.4x.

Wenjie Sha (Rutgers University) then presented ``Social Vehicle Navigation: 
Integrating Shared Driving Experience into Vehicle Navigation''. Sha described 
their system `NaviTweet', a social navigation application, which allows drivers 
to post or listen to traffic-related voice tweets grouped in a vehicular social 
network (VSN, a social network of drivers who travel the same routes or to the 
same destination). Voice tweets are periodically formed into digests by the 
NaviTweet server and then sent to driver's mobile device for playback. Their 
prototype implementation uses OpenStreetMap for mapping data, Google Map for VSN 
management and Android speech recognition. Whilst the audio nature of the tweets 
was intended to reduce cognitive overload, audience questions highlighted the 
potential for background noise to make interpretation challenging and even for 
the opportunity to abuse the system for transmitting negative, `road rage', 
messages.

The final paper, ``Quantifying the Potential of Ride-Sharing using Call 
Description Records'', was presented by Blerim Cici (University of California). 
In this work, the authors attempt to quantify the upper bound on ride-sharing 
effectiveness by considering human mobility patterns in Madrid, Spain, using a
3-month call description record (CDR) dataset. The authors assumed each traveller 
had a car capacity of four and by matched users whose home and work locations 
were close to each other. By permitting short detours to pickup\slash drop-offs,
analysis shows that ride-sharing has the potential to eliminate ~50\% of car 
journeys. Questions and discussion following the presentation focussed on 
incentives to share (\eg{} using the social graph), increasing opportunity by 
allowing passengers divide their route between cars, and opportunities for 
analysing CDRs to improve public transportation.