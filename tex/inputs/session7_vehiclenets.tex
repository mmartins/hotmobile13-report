%!TEX root = ../hotmobile13-report.tex
\section{Vehicular Networking and Transportation}
\label{sec:vehiclenets}

The final session, on ``Vehicular Networking and Transportation'' was chaired by 
Lin Zhong and was opened with a presentation by Tan Zhang (University of 
Wisconsin Madison), who presented ``Scout: An Asymmetric Vehicular Network Design 
over TV Whitespaces''. Motivated by an increasing trend towards in-vehicle 
Internet connectivity (\eg{} for navigation, safety and entertainment), their 
work provides an architecture for vehicular networking that is designed to 
provide good coverage and throughput. By leveraging locally available television 
frequencies (white spaces), a small number of base stations can cover a 
significant area at low cost (due to their good propagation characteristics).
Legal restrictions on power transmission suggest an asynchronous design and so 
the white space is used as a downlink whilst cellular 3G is used for the uplink.
By using two radios on each vehicle (the target application is a set of 250 city 
buses), the front (scout) can monitor changing channel characteristics (\eg{} as 
the vehicle moves behind a building) and inform the base station which can then 
use optimum parameters for communication with the rear radio. Initial evaluation 
results indicate that using this architecture can improve base station coverage 
by 4? coverage improvement and achieve a throughput gain of 1.4?.

The second paper in the session, ``Social Vehicle Navigation: Integrating Shared 
Driving Experience into Vehicle Navigation'', was presented by Wenjie Sha 
(Rutgers University). Sha describes their system `NaviTweet', a social vehicle
navigation application, which allows drivers to post or listen to traffic-related 
voice tweets grouped in a vehicular social network (VSN, a social network of 
drivers who travel the same routes or to the same destination). Voice tweets are 
periodically formed into digests (\eg{} grouped by road segment) by the NaviTweet 
server. The short audio clips are designed to convey recent, relevant information 
in such a way as to prevent driver cognitive overload. At the driver's mobile 
device tweet dig rest are pruned (based on trajectory) and then played (ordered 
by distance the from driver). Sha and colleagues have a prototype implementation
using OpenStreetMap for mapping data, Google Map for VSN management and Android 
speech recognition.

Finally, Blerim Cici (University of California) presented ``Quantifying the 
Potential of Ride-Sharing using Call Description Records'', work done in 
partnership with Telefonica Research. In Cici et al's work, the authors attempt 
to quantify the upper bound on ride-sharing effectiveness. They examine human 
mobility patterns within the city of Madrid, Spain, using a call description 
records (CDR) dataset.